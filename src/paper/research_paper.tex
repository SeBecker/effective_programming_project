\documentclass[11pt, a4paper, leqno]{article}
\usepackage{a4wide}
\usepackage[T1]{fontenc}
\usepackage[utf8]{inputenc}
\usepackage{float, afterpage, rotating, graphicx}
\usepackage{epstopdf}
\usepackage{longtable, booktabs, tabularx}
\usepackage{fancyvrb, moreverb, relsize}
\usepackage{eurosym, calc, chngcntr}
\usepackage{amsmath, amssymb, amsfonts, amsthm, bm}
\usepackage{caption}
\usepackage{mdwlist}
\usepackage{xfrac}
\usepackage{setspace}
\usepackage{xcolor}
\usepackage{mathptmx}
\usepackage[a4paper]{geometry}
\geometry{verbose,tmargin=2cm,bmargin=2cm,lmargin=3cm,rmargin=2cm}
\usepackage{textcomp}
\usepackage{esint}
\usepackage{nomencl}
\usepackage{titlesec}
\usepackage{geometry}
\usepackage{multirow}
\usepackage{adjustbox}
\usepackage{array}
\usepackage{fp}
\usepackage{numprint}
\usepackage{expl3}
\usepackage[section]{placeins}


% \usepackage{pdf14} % Enable for Manuscriptcentral -- can't handle pdf 1.5
% \usepackage{endfloat} % Enable to move tables / figures to the end. Useful for some submissions.

\usepackage[
    natbib=true,
    bibencoding=inputenc,
    bibstyle=authoryear-ibid,
    citestyle=authoryear-comp,
    maxcitenames=3,
    maxbibnames=10,
    useprefix=false,
    sortcites=true,
    backend=bibtex
]{biblatex}
\AtBeginDocument{\toggletrue{blx@useprefix}}
\AtBeginBibliography{\togglefalse{blx@useprefix}}
\setlength{\bibitemsep}{1.5ex}
\addbibresource{refs.bib}

\usepackage[unicode=true]{hyperref}
\hypersetup{
    colorlinks=true,
    linkcolor=black,
    anchorcolor=black,
    citecolor=black,
    filecolor=black,
    menucolor=black,
    runcolor=black,
    urlcolor=black
}


\widowpenalty=10000
\clubpenalty=10000

\setlength{\parskip}{1ex}
\setlength{\parindent}{0ex}
\setstretch{1.5}


\begin{document}


\title{Replication - Income and Democracy by Acemoglu, Johnson, Robinson and Yared
\thanks{Sebastian Becker: University of Bonn, Address. \href{mailto:x@y.z} {\nolinkurl{x [at] y [dot] z}}, tel.~+00000.}
% subtitle:
% \\[1ex]
% \large Subtitle here
}

\author{Sebastian Becker
% \\[1ex]
% Additional authors here
}

\date{
\today}

\maketitle
\clearpage

\section{Introduction} % (fold)
\label{sec:introduction}
The following document includes all figures and tables from \citet{Acemoglu1} and works with a template by \citet{GaudeckerEconProjectTemplates}.

Does economic development influences the political situation of countries such that they become more democratic over time? A quick look on the relationship of income per capita and measures of democracy (Figure1) seems to support this assumption. Today the most developed countries are democratic while many of the most poorest have autocratic or dictatorial political systems. Furthermore the process of democratization in Europe and America came together with the extreme increasing of growth rates at the beginning of the 19th century. But is there a statistical evidence that this cross section correlation between income growth and political changes is based on an causal effect? There are many difficulties if you want to analyze this relationship. First there could be reverse causality such that a more democratic system provides its citizens a higher income. Second there is a large potential for omitted variables because it doesn`t seem clear what factors determine the nature of a political system. \\
Acemoglu, Johnson, Robinson and Yared try to avoid this biases through the implementation of a fixed effect analysis and an instrumental variable approach. For increasing the robustness of their findings they use different measures of democracy, two different variables to instrument income and integrate different time periods in their analysis.
However, the authors state that their is no significant evidence for a causal relationship neither in the fixed effect nor in the 2SLS analysis. Finally the autohors distiguish that even if there is no simple causal link between political and economical development, they are jointly influenced by various factors. Furthermore the positive association over the past 500 years support the hypothesis of divergent long term development paths (Table 10/11).



% section introduction (end)



\newpage{}

\section{Figures}
\label{sec:Figures}


\begin{figure}[!htbp]
	\includegraphics{../../out/figures/F1.eps}
\end{figure}


\begin{figure}[!htbp]
	\includegraphics{../../out/figures/F2.eps}
\end{figure}
\begin{figure}[!htbp]
    \includegraphics{../../out/figures/F3.eps}
\end{figure}
\begin{figure}[!htbp]
    \includegraphics{../../out/figures/F4.eps}
\end{figure}
\begin{figure}[!htbp]
    \includegraphics{../../out/figures/F5.eps}
\end{figure}
\begin{figure}[!htbp]
    \includegraphics{../../out/figures/F6.eps}
\end{figure}


\clearpage

\section{Tables}

\subsection{Fixed effects analysis}
\input{../../out/tables/FHI.tex}

\input{../../out/tables/POL.tex}

\clearpage

\subsection{Robustness Check}

\input{../../out/tables/FHI_robust.tex}

\input{../../out/tables/POL_robust.tex}
\clearpage


\subsection{Fixed effects results using 2SLS}

\input{../../out/tables/FHI_nsave_IVtable.tex}

\input{../../out/tables/FHI_worldincome_IVtable.tex}

\clearpage

\input{../../out/tables/POL_nsave_IVtable.tex}

\input{../../out/tables/POL_worldincome_IVtable.tex}

\clearpage

\subsection{Fixed effects analysis in the long run}

\input{../../out/tables/long_run_table.tex}

\clearpage

\subsection{Democracy in the very long run}

\input{../../out/tables/very_long_A.tex}

\input{../../out/tables/very_long_B.tex}

\clearpage
\setstretch{1}
\printbibliography
\setstretch{1.5}

%\appendix
%\counterwithin{table}{section}
%\counterwithin{figure}{section}

\end{document}
